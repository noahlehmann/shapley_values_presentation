\ac{ML} has come a long way in the last few years and starts to unfold possibilities that nobody could think of just recently.
Distributed learning for instance makes it possible to train highly sophisticated models while still preserving the data's
owners privacy rights.
On the other hand, \acp{AI} have come far enough to beat human professionals in all kinds of quests, making them more powerful -
even if only highly specialised.

The challenge now is to find ways to actually leverage use from these powerful mechanisms, preferably in providing humans
assistance in situations where it might not have been possible without \acp{AI}.
One way is to use this computed assistance in situations where humans tend to isolate themselves, especially when they
start to develop depressive tendencies.
It is not just the isolation, in some cases the affected people don't recognize their tendencies themselves - thus not
seeing the need for help from a professional.

In these cases, \emph{emotion analysis} may come in handy, as the affected persons texts, posts on social media or general communication
can be analysed and later categorised, thus recognizing patterns early and then acting in a defined way to take the first
steps in helping the user to overcome his or her problems.

The emotion analysis described here comes with multiple caveats, mainly in the concern of data privacy and regulations concerning
the analysis of the written data and in the way, the models recognize patterns and thus decide on whether they have spotted
a pattern of depression.

As this study paper is only trying to show the potential of the idea, the data privacy regulations and all other aspects
to be considered when handling and analysing user data, will not be discussed.
The consideration of how the models make their decisions and what makes them decide as they do will be a main part of the
analysis though and will be discussed thoroughly later in the paper.

\subsection{Emotion Analysis}

In order to see the big picture in the explanation of the models decision, it is important to have a test scenario to which
one can refer.
As explained, \acp{AI} tend to take more and more important tasks, thus making it more important to understand their decision
flows.

One sensitive topic is emotion analysis, where a user could potentially be classified as depressive in how they write and speak.
Emotion analysis could gather the users outputs in multiple formats, analyse them in bulks and recognize major changes in
their behaviour.
This can be used in different ways, on the one hand recognizing texts that should be banned from social media, such as hate speech
and similar types of texts.
On the other hand this can, as described before, also recognize depressive tendencies, thus providing help in early stages.

Chapter~\ref{sec:sadness} will discuss such an approach to \emph{sadness}-detection, followed by a technique, that provides insights
in how the decisions are made.

\subsection{Explainable \ac{AI}}

As mentioned before, \acp{AI} get more and more complex.
Especially looking at classifiers like
